\documentclass[10pt]{extarticle}
\usepackage[utf8]{inputenc}
\usepackage[english, ngerman]{babel}
\usepackage{listings}
\usepackage{geometry}
\usepackage{booktabs}
\usepackage[fleqn]{amsmath}
\usepackage{amssymb}
\usepackage{multicol}
\usepackage{multirow}
\usepackage{graphicx}
\usepackage{breqn}
\usepackage{tabularx}
\usepackage{titlesec}
\usepackage{lscape}
\usepackage{blindtext}
\usepackage{times}
\usepackage{authblk}
\usepackage{xcolor}

\setlength{\mathindent}{0mm}
\setlength{\columnsep}{0.8cm}
\setlength\parindent{0pt} 	


 \geometry{
 a4paper,
 total={190mm,270mm},
 left=20mm,
 right=20mm,
 top=20mm,
 bottom=20mm,
 }
 
 \titleformat{\title}
 {\normalfont\fontsize{11}{10}\bfseries}{\thesection}{1em}{} 
 
 
 \titleformat{\section}
  {\normalfont\fontsize{11}{10}\bfseries}{\thesection}{1em}{} 


\titleformat{\subsection}
  {\normalfont\fontsize{10}{5}\bfseries}{\thesection}{1em}{} 

\renewcommand{\arraystretch}{1.3}

%Path relative to the .tex file containing the \includegraphics command
\graphicspath{ {figures/} }

\definecolor{light-gray}{gray}{0.90}

\date{}

\makeatletter


\usepackage{ragged2e}
\newcolumntype{L}[1]{>{\raggedright\arraybackslash}p{#1}}
\newcolumntype{C}[1]{>{\centering\arraybackslash}p{#1}}
\newcolumntype{R}[1]{>{\raggedleft\arraybackslash}p{#1}}
\newcolumntype{J}[1]{>{\justifying\arraybackslash}p{#1}}

\renewenvironment{abstract}
 {\par\noindent\ignorespaces}
 {\par\medskip}

\renewcommand{\maketitle}{\setlength{\parindent}{0pt}
\begin{flushleft}
  \huge\@title
  \par\vspace{1cm}
  \normalsize\@author
  \vspace{1cm}  
  
  \hrule
  \begin{minipage}[t]{\textwidth}
    \begin{minipage}[t]{0.25\textwidth}
    \vspace{0.3cm}
	Protokoll-Nr. 4 
	Index 4
    \vspace{0.3cm}
	\hrule
	\vspace{0.3cm}
	\date{\today}
	\vspace{0.3cm}
	\end{minipage}
	\begin{minipage}[t]{0.05\textwidth}
	 \hfill
	\end{minipage}
	\begin{minipage}[t]{0.70\textwidth}
	\begin{flushleft}    
	\vspace{0.3cm}
	\textbf{ORGANISATION IM TEAM}
	\vspace{0.3cm}
	
\begin{tabularx}{11cm} { 
   L{5cm}
   L{6cm}
  }
 \toprule
 Sitzung: & 29.06.2022 20.00 - 21.00 Uhr\\
 Teilnehmer & Janick (JZ) \\
 			& Henry (HG) \\
 			& Luzi (LS) \\
 			& Lukas (LR) \\
            & Patrick (PS)\\
 Meeting & Google Meet \\
 \end{tabularx}	
\begin{abstract}
\end{abstract}
\end{flushleft}
    \vspace{0.3cm}
	\end{minipage}
  \end{minipage}
  
  \hrule
\end{flushleft}
\vspace{1cm}

}
\makeatother

%%%%%%%%%%%%%%%%%%%%%%%%%%%%%%%%%%%%%%%%%%%%%%%%%%%%%%%%%%%%%%%

\begin{document}


\title{Sitzungsprotokoll 4}
\author{Janick Zehnder}
\maketitle


\colorbox{light-gray}{\begin{minipage}{17cm}
\vspace{0.25cm}
\section*{Ziele}
\begin{itemize}
\item Sommferferien 2022 und nächste Meetingdaten
\item Latex Installations-Check
\item Git Status
\item Zwei begründete Ideenfavoriten
\item komplexe vs. einfache Produktideen
\end{itemize}
\vspace{0.25cm}
\end{minipage}}



\section{Sommerferien 2022 und nächste Meetingdaten}
\textbf{Sommerferien 2022:} 25. Juli 2022XXX.\\
\textbf{Nächste Meetingdaten} XXX\\

\section{Latx Installations-Check}
Hat bei jedem die Installation geklappt?\\
Link im IT Ordner abgelegt?\\ Ok.

\section{Git Status}
\textbf{LS:}\\
Einfürhung Git
nicht nur für Software, auch für Protokoll möglich
Tags
Test mit Protokoll
GUI sourcetree herunterladen
er macht Anleitung
admin rechte für luzi
Buch von Luzi für Einführung

\section{Zwei begründete Ideenfavoriten}
\textbf{JZ:}\\ 1) 2) komplexe Systeme.
\textbf{HG:}\\ komplettes Produkt da alle Disziplinen enthalten. kein Gadget. nicht nur nice to have. 1) mobile Einsatzzentrale / Wohnunterkunft. kleine Häuse (Tiny Haus), LED Anzeige obendrauf, Impfbox 1 / Auskunftszentrale / Wohnunterkunft, Läute aussen, Gebäudeautomation, modulare Einrichtung, läuten funktionsspezifisch, Vernetzung, Kopplung mit BUS koppelbare Einheiten. Grösse? 2) Druckkonturkissen, Erleichterung für Patienten, Palette erweiterbar zb für Bettlieger wenn es gut ankommt, Akku oder Netzbetrieben.
\textbf{LS:}\\ 2) Kehricht mechanisch elektrisch verpresst, verbindet, für Anwender schmutzfrei, für jeden Haushalt, Restaurant, kein Need für grosse Presse, in Keller oder unter Lavabo, "Kleinpresse" für Haushalt. Wieso? mech. el. saueber lösen, viele Kunden.
1) Babyschaukel mit Schwingungsaufnahme, vertikal und horizontal, Soundanlage zur Beruhigung, Pulsuhr, Schlafrhytmus, Temperaturmessung optisch. Fokus war auf komplettes Produkt. Viel zu lernen, komplex aber möglich mit unseren Leuten. 
\textbf{LR:}\\ 1) smart wheel: alles drin, viel software, algorithmen, hardware, statik, komplexe Sache interdisziplinär 2) Car communication, gadget, nice to have, beifahrer kann mit anderem fahrer kommunizieren.
\textbf{PS:}\\

Anleitung Luzi
GIT Gui vorbereiten mit Serverzugriff
individuelle Recherche(start)

LR: Arbeiten abmachen und es wird erledigt. Überlegungen zu Skalierbarkeit geschätzt.
HG: GIT cool, gute Grundbasis. Freut sich auf intensivieren.
LS: danke für Feedback GIT. es läuft, erstesmal etwas gemacht. gut dran zu sein, nicht nichts tun wie CHEFS! TUN, erfolg hat nur 3 Buchstaben. Lob an kreativen Henry.
JZ: 

\section{komplexe vs. einfache Produktideen}

\vspace{2cm}
\section*{Nächste Sitzung}
29.06.2022 20:00 Uhr Google Meet

\section*{Pendenzen}
\begin{tabularx}{17cm} { 
   C{1.0cm}
   L{7cm}
   C{2cm}
   C{2cm}
   C{2cm}
  }
 \toprule
Pendenz & Beschreibung & Zuständig & Datum & Status \\
\midrule
01	& Anleitung zu Download TexMaker/Latex  & PS & 29.06.2022 & offen\\
02	& Git aufsetzen & LS & 29.06.2022 & offen\\
03 	& Zwei Themen von allen Ideen aussuchen und begründen & alle & 29.06.2022 & offen\\
04  & Gedanken machen welche Richtung man einschlagen will: Komplexes Produkt oder einfachere Strukturen & alle & 29.06.2022 & offen\\ 
\bottomrule
 \end{tabularx}	

\end{document}

