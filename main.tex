\documentclass[10pt]{extarticle}
\usepackage[utf8]{inputenc}
\usepackage[english, ngerman]{babel}
\usepackage{listings}
\usepackage{geometry}
\usepackage{booktabs}
\usepackage[fleqn]{amsmath}
\usepackage{amssymb}
\usepackage{multicol}
\usepackage{multirow}
\usepackage{graphicx}
\usepackage{breqn}
\usepackage{tabularx}
\usepackage{titlesec}
\usepackage{lscape}
\usepackage{blindtext}
\usepackage{times}
\usepackage{authblk}
\usepackage{xcolor}

\setlength{\mathindent}{0mm}
\setlength{\columnsep}{0.8cm}
\setlength\parindent{0pt} 	


 \geometry{
 a4paper,
 total={190mm,270mm},
 left=20mm,
 right=20mm,
 top=20mm,
 bottom=20mm,
 }
 
 \titleformat{\title}
 {\normalfont\fontsize{11}{10}\bfseries}{\thesection}{1em}{} 
 
 
 \titleformat{\section}
  {\normalfont\fontsize{11}{10}\bfseries}{\thesection}{1em}{} 


\titleformat{\subsection}
  {\normalfont\fontsize{10}{5}\bfseries}{\thesection}{1em}{} 

\renewcommand{\arraystretch}{1.3}

%Path relative to the .tex file containing the \includegraphics command
\graphicspath{ {figures/} }

\definecolor{light-gray}{gray}{0.90}

\date{}

\makeatletter


\usepackage{ragged2e}
\newcolumntype{L}[1]{>{\raggedright\arraybackslash}p{#1}}
\newcolumntype{C}[1]{>{\centering\arraybackslash}p{#1}}
\newcolumntype{R}[1]{>{\raggedleft\arraybackslash}p{#1}}
\newcolumntype{J}[1]{>{\justifying\arraybackslash}p{#1}}

\renewenvironment{abstract}
 {\par\noindent\ignorespaces}
 {\par\medskip}

\renewcommand{\maketitle}{\setlength{\parindent}{0pt}
\begin{flushleft}
  \huge\@title
  \par\vspace{1cm}
  \normalsize\@author
  \vspace{1cm}  
  
  \hrule
  \begin{minipage}[t]{\textwidth}
    \begin{minipage}[t]{0.25\textwidth}
    \vspace{0.3cm}
	Protokoll-Nr. 5 
	Index 5
    \vspace{0.3cm}
	\hrule
	\vspace{0.3cm}
	\date{\today}
	\vspace{0.3cm}
	\end{minipage}
	\begin{minipage}[t]{0.05\textwidth}
	 \hfill
	\end{minipage}
	\begin{minipage}[t]{0.70\textwidth}
	\begin{flushleft}    
	\vspace{0.3cm}
	\textbf{ORGANISATION IM TEAM}
	\vspace{0.3cm}
	
\begin{tabularx}{11cm} { 
   L{5cm}
   L{6cm}
  }
 \toprule
 Sitzung: & 25.07.2022 20.00 - 21.00 Uhr\\
 Teilnehmer & Janick (JZ) \\
 			& Henry (HG) \\
 			& Luzi (LS) \\
 			& Lukas (LR) \\
            & Patrick (PS)\\
 Meeting & Google Meet \\
 \end{tabularx}	
\begin{abstract}
\end{abstract}
\end{flushleft}
    \vspace{0.3cm}
	\end{minipage}
  \end{minipage}
  
  \hrule
\end{flushleft}
\vspace{1cm}

}
\makeatother

%%%%%%%%%%%%%%%%%%%%%%%%%%%%%%%%%%%%%%%%%%%%%%%%%%%%%%%%%%%%%%%

\begin{document}


\title{Sitzungsprotokoll 5}
\author{Janick Zehnder}
\maketitle


\colorbox{light-gray}{\begin{minipage}{17cm}
\vspace{0.25cm}
\section*{Ziele}
\begin{itemize}
\item Follow-Up Git Installation und Einführung
\item Individueller Einblick in Recherche Ideenfavorit Nr. 1
\end{itemize}
\vspace{0.25cm}
\end{minipage}}



\section{Follow-Up Git Installation und Einführung}
\textbf{PS/LS:} Patrick Kontakt an Luzi wegen NAS / GIT.\\

\section{Individueller Einblick in Recherche Ideenfavorit Nr. 1}
\textbf{JZ:} SmartWheel - Fokus Patentrecherche\\
\textbf{PS:} Polyloft. Aus Holz. Nicht innovativ. Coole Ideen auf Pinterest. Effiziente Raumgestaltung: Bett, Sofa, Küche... Einschub Janick mit Trendlounge.\\
\textbf{LR:} Babyschaukel - Jeder ist betroffen / mit wenig Aufwand weit kommen / 85000 Babys in der Schweiz / viele Schreibabys (20 Prozent) / wichtig ist Funktion, Sicherheit, Vertrauen, Optik, Haptik, ökologisch, Zusatzfunktionen. Konkurenzprdoukte gezeigt. Kunde macht selber Werbung.\\
\textbf{HG:} Flexbox: ein Produkt zwischen Tiny-Haus und Container-Haus. Man kann sie brauchen für alles. Koppelbar. Farbige Panels zur visuellen Unterstützung für Notsituationen. Stapelbar und reihbar. Viele bauen Tiny-Häuser und habe keine grosse ERfahrung darin. Markt der Container geht durch die Decke seit Pandemie. Einsatz für Camping bei OpenAirs. Pandemiesituationen. Mobilität. Modularität. Kein expliziter Anbieter von Flex Boxen: nichts für modulare Zwecke. 15000 bis 100000 CHF.\\
\textbf{LS:} AGV - Maxon beliefert diese. Elektrohubwagen. Bohle Transportwagen. Anhängerlenkrad. Schlanser - Antrieb auf Gummirad. Rangierhilfe. Bestehendes nehmen und besser machen. Scheren-Hebebühnen: geringe Traglast bei hohem Eigengewicht.\\

\vspace{2cm}
\section*{Nächste Sitzung}
Idee von JZ: physisches Treffen mit 5-10min Flash-Präsentation über erweiterte Recherche zur favorisierten Idee.\\

Montag, 08.08.2022, 20:00 Uhr, online Meeting

\section*{Pendenzen}
\begin{tabularx}{17cm} { 
   C{1.0cm}
   L{7cm}
   C{2cm}
   C{2cm}
   C{2cm}
  }
 \toprule
Pendenz & Beschreibung & Zuständig & Datum & Status \\
\midrule
01	& GIT Follow-Up  & LS & 08.08.2022 & offen\\
02	& GIT Kontakt von PS an LS & PS & 08.08.2022 & offen\\
03	& Ausfüllen Doodle physisches Meeting  & alle & 08.08.2022 & offen\\
04	& Vereinbarung Präsentationsinhalt & alle & 08.08.2022 & offen\\
\bottomrule
 \end{tabularx}	

\end{document}

