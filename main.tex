\documentclass[10pt]{extarticle}
\usepackage[utf8]{inputenc}
\usepackage[english, ngerman]{babel}
\usepackage{listings}
\usepackage{geometry}
\usepackage{booktabs}
\usepackage[fleqn]{amsmath}
\usepackage{amssymb}
\usepackage{multicol}
\usepackage{multirow}
\usepackage{graphicx}
\usepackage{breqn}
\usepackage{tabularx}
\usepackage{titlesec}
\usepackage{lscape}
\usepackage{blindtext}
\usepackage{times}
\usepackage{authblk}
\usepackage{xcolor}

\setlength{\mathindent}{0mm}
\setlength{\columnsep}{0.8cm}
\setlength\parindent{0pt} 	


 \geometry{
 a4paper,
 total={190mm,270mm},
 left=20mm,
 right=20mm,
 top=20mm,
 bottom=20mm,
 }
 
 \titleformat{\title}
 {\normalfont\fontsize{11}{10}\bfseries}{\thesection}{1em}{} 
 
 
 \titleformat{\section}
  {\normalfont\fontsize{11}{10}\bfseries}{\thesection}{1em}{} 


\titleformat{\subsection}
  {\normalfont\fontsize{10}{5}\bfseries}{\thesection}{1em}{} 

\renewcommand{\arraystretch}{1.3}

%Path relative to the .tex file containing the \includegraphics command
\graphicspath{ {figures/} }

\definecolor{light-gray}{gray}{0.90}

\date{}

\makeatletter


\usepackage{ragged2e}
\newcolumntype{L}[1]{>{\raggedright\arraybackslash}p{#1}}
\newcolumntype{C}[1]{>{\centering\arraybackslash}p{#1}}
\newcolumntype{R}[1]{>{\raggedleft\arraybackslash}p{#1}}
\newcolumntype{J}[1]{>{\justifying\arraybackslash}p{#1}}

\renewenvironment{abstract}
 {\par\noindent\ignorespaces}
 {\par\medskip}

\renewcommand{\maketitle}{\setlength{\parindent}{0pt}
\begin{flushleft}
  \huge\@title
  \par\vspace{1cm}
  \normalsize\@author
  \vspace{1cm}  
  
  \hrule
  \begin{minipage}[t]{\textwidth}
    \begin{minipage}[t]{0.25\textwidth}
    \vspace{0.3cm}
	Protokoll-Nr. 6 
	Index 1
    \vspace{0.3cm}
	\hrule
	\vspace{0.3cm}
	\date{\today}
	\vspace{0.3cm}
	\end{minipage}
	\begin{minipage}[t]{0.05\textwidth}
	 \hfill
	\end{minipage}
	\begin{minipage}[t]{0.70\textwidth}
	\begin{flushleft}    
	\vspace{0.3cm}
	\textbf{VORGEHEN THEMENRECHERCHE}
	\vspace{0.3cm}
	
\begin{tabularx}{11cm} { 
   L{5cm}
   L{6cm}
  }
 \toprule
 Sitzung: & 08.08.2022 20.00 - 21.00 Uhr\\
 Teilnehmer: & Janick (JZ) \\
 			& Henry (HG) \\
 			& Lukas (LR) \\
            & Patrick (PS)\\
Entschuldigt: & Luzi (LS) \\
 Meeting: & Google Meet \\
 \end{tabularx}	
\begin{abstract}
\end{abstract}
\end{flushleft}
    \vspace{0.3cm}
	\end{minipage}
  \end{minipage}
  
  \hrule
\end{flushleft}
\vspace{1cm}

}
\makeatother

%%%%%%%%%%%%%%%%%%%%%%%%%%%%%%%%%%%%%%%%%%%%%%%%%%%%%%%%%%%%%%%

\begin{document}


\title{Sitzungsprotokoll 6}
\author{Patrick Schulthess}
\maketitle


\colorbox{light-gray}{\begin{minipage}{17cm}
\vspace{0.25cm}
\section*{Ziele}
\begin{itemize}
\item Organisation NAS
\item Festlegen der für die Themenrecherche zu berücksichtigenden Punkte
\item Terminfindung Treffen
\end{itemize}
\vspace{0.25cm}
\end{minipage}}

\section{Organisation NAS}
Der Kontakt von PS bestätigte, dass eine Installation von git auf einem Synology NAS möglich ist und von einer Gruppe genutzt werden kann. PS schafft sich ein Synology NAS an und kümmert sich um eine Lösung für die Zusammenarbeit mit git. In der Zwischenzeit kann die Lösung mit dem öffentlichen git genutzt werden. Ziel ist, dass eine definitive Lösung in ungefähr zwei Monaten vorliegt.

\section{Themenrecherche}
Aktuell stehen verschiedene Themen für die weitere Bearbeitung zur Diskussion. Um schlussendlich eine möglichst gut begründbare Themenwahl treffen zu können, soll zu jedem Thema eine vertiefte Recherche durchgeführt werden. Ziel ist, dass die Gruppe sich über den Umfang, die Herausforderungen, Chancen und Risiken sämtlicher Themen im klaren ist. Damit die Grundlagen zu jedem Thema in ähnlicher Form erarbeitet werden hat LR vorgeschlagen, dass folgende Punkte behandelt werden sollten:

\begin{itemize}
\item Produktbeschrieb
\item Anforderungen
\item Marktgrösse
\item Konkurrenz \& Preise
\item Ideen für Weiterführung
\end{itemize}

Im ersten Schritt erfolgt die Themenrecherche in Form eines Brainstormings wobei versucht werden soll, alle oben genannten Punkte möglichst gut abzudecken. JZ wird als zusätzliche Inspiration ein PDF-Dokument namens \textit{Business Model Canvas} auf dem NAS ablegen. Das Brainstorming kann auch direkt in dieser Vorlage eingetragen werden. 
\par
\vspace{0.2cm}
Die Recherche zu einigen der genannten Punkte wird vermutlich nicht einfach sein. Als mögliche Quellen für Informationen wurden das Bundesamt für Statistik oder die bereits diskutierte Patentrecherche genannt. 
\par\vspace{0.2cm}
LR schlug vor, dass nachdem die Themenrecherche abgeschlossen ist, eine Rochade stattfindet und jeder sein Thema an einen Kollegen abgibt, der die Recherche dann vertieft. Dadurch dass mehrere Personen ein Thema bearbeiten, können die Stärken des Teams besser genutzt werden. Die Teilnehmer waren mit diesem Vorschlag einverstanden. Das Vorgehen wird im weiteren Projektverlauf entsprechend umgesetzt.

\section{Terminfindung}
Das nächste Treffen der Gruppe wurde festgelegt (Details siehe nächste Sitzung).

\par
\vspace{0.5cm}
\textit{Bemerkung: Eine Reservationsanfrage beim Restaurant Modomio wurde getätigt. Sobald die Reservation bestätigt wird, werden die Teilnehmer informiert und das Protokoll entsprechend ergänzt. Falls die Reservation nicht bestätigt werden sollte, tätigt PS eine weitere Anfrage beim Restaurant La Baracca und informiert die Gruppe bis spätestens Ende KW32}

\vspace{2cm}
\section*{Nächste Sitzung}

Freitag, 02.09.2022, 18:30 Uhr, Restaurant Modomio, St Karliquai 9, 6004 Luzern

\section*{Pendenzen}
\begin{tabularx}{17cm} { 
   C{1.0cm}
   L{7cm}
   C{2cm}
   C{2cm}
   C{2cm}
  }
 \toprule
Pendenz & Beschreibung & Zuständig & Datum & Status \\
\midrule
01	& GIT Follow-Up  & LS & 08.08.2022 & offen\\
02	& GIT Kontakt von PS an LS & PS & 08.08.2022 & erledigt\\
03	& Ausfüllen Doodle physisches Meeting  & alle & 08.08.2022 & erledigt\\
04	& Vereinbarung Präsentationsinhalt & alle & 08.08.2022 & erledigt\\
05  & Anschaffen NAS und Installation git & PS & 02.09.2022 & offen \\
06  & Hochladen Business Model Canvas & JZ & 02.09.2022 & offen\\
07  & Brainstorming zum Thema & alle & 02.09.2022 & offen\\
\bottomrule
\end{tabularx}	
\par
\vspace{0.5cm}
Erledigte Pendenzen werden einmal als erledigt markiert und anschliessend gelöscht.
\end{document}

