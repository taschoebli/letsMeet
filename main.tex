\documentclass[10pt]{extarticle}
\usepackage[utf8]{inputenc}
\usepackage[english, ngerman]{babel}
\usepackage{listings}
\usepackage{geometry}
\usepackage{booktabs}
\usepackage[fleqn]{amsmath}
\usepackage{amssymb}
\usepackage{multicol}
\usepackage{multirow}
\usepackage{graphicx}
\usepackage{breqn}
\usepackage{tabularx}
\usepackage{titlesec}
\usepackage{lscape}
\usepackage{blindtext}
\usepackage{times}
\usepackage{authblk}
\usepackage{xcolor}

\setlength{\mathindent}{0mm}
\setlength{\columnsep}{0.8cm}
\setlength\parindent{0pt} 	


 \geometry{
 a4paper,
 total={190mm,270mm},
 left=20mm,
 right=20mm,
 top=20mm,
 bottom=20mm,
 }
 
 \titleformat{\title}
 {\normalfont\fontsize{11}{10}\bfseries}{\thesection}{1em}{} 
 
 
 \titleformat{\section}
  {\normalfont\fontsize{11}{10}\bfseries}{\thesection}{1em}{} 


\titleformat{\subsection}
  {\normalfont\fontsize{10}{5}\bfseries}{\thesection}{1em}{} 

\renewcommand{\arraystretch}{1.3}

%Path relative to the .tex file containing the \includegraphics command
\graphicspath{ {figures/} }

\definecolor{light-gray}{gray}{0.90}

\date{}

\makeatletter


\usepackage{ragged2e}
\newcolumntype{L}[1]{>{\raggedright\arraybackslash}p{#1}}
\newcolumntype{C}[1]{>{\centering\arraybackslash}p{#1}}
\newcolumntype{R}[1]{>{\raggedleft\arraybackslash}p{#1}}
\newcolumntype{J}[1]{>{\justifying\arraybackslash}p{#1}}

\renewenvironment{abstract}
 {\par\noindent\ignorespaces}
 {\par\medskip}

\renewcommand{\maketitle}{\setlength{\parindent}{0pt}
\begin{flushleft}
  \huge\@title
  \par\vspace{1cm}
  \normalsize\@author
  \vspace{1cm}  
  
  \hrule
  \begin{minipage}[t]{\textwidth}
    \begin{minipage}[t]{0.25\textwidth}
    \vspace{0.3cm}
	Protokoll-Nr. 6 
	Index 1
    \vspace{0.3cm}
	\hrule
	\vspace{0.3cm}
	\date{\today}
	\vspace{0.3cm}
	\end{minipage}
	\begin{minipage}[t]{0.05\textwidth}
	 \hfill
	\end{minipage}
	\begin{minipage}[t]{0.70\textwidth}
	\begin{flushleft}    
	\vspace{0.3cm}
	\textbf{VORGEHEN THEMENRECHERCHE}
	\vspace{0.3cm}
	
\begin{tabularx}{11cm} { 
   L{5cm}
   L{6cm}
  }
 \toprule
 Sitzung: & 02.09.2022 21.00 - 22.00 Uhr\\
 Teilnehmer: & Janick (JZ) \\
 			& Henry (HG) \\
 			& Lukas (LR) \\
            & Patrick (PS)\\
			& Luzi (LS) \\
 Meeting: & Restaurant Mondino Luzern \\
 \end{tabularx}	
\begin{abstract}
\end{abstract}
\end{flushleft}
    \vspace{0.3cm}
	\end{minipage}
  \end{minipage}
  
  \hrule
\end{flushleft}
\vspace{1cm}

}
\makeatother

%%%%%%%%%%%%%%%%%%%%%%%%%%%%%%%%%%%%%%%%%%%%%%%%%%%%%%%%%%%%%%%

\begin{document}


\title{Sitzungsprotokoll 7}
\author{Luzi Schöb}
\maketitle


\colorbox{light-gray}{\begin{minipage}{17cm}
\vspace{0.25cm}
\section*{Ziele}
\begin{itemize}
\item Fokus Idee
\item Auswahl Idee
\item Weiteres Vorgehen Idee
\end{itemize}
\vspace{0.25cm}
\end{minipage}}

\section{Fokus Idee}
HG hat die FlexBox anhand des Business Model Canvas als Erster vorgestellt und gut verkauft. Die Modularität liegt im Vordergrund. Hauptkunden sind Bund und Hilfsorganisationen. Siehe BMC HG. PS hat ebenfalls zum TinyHouse recherchiert mit mehr Fokus auf Ästhetik und Privatkunden. LR (Babyschaukel) hat erwähnt, dass wir den Kunden von Anfang an an der Entwicklung dabei haben sollen (zumindest ihm das Gefühl geben). LR erwähnt, dass eine Zwischennutzung (z.B. für Festivals) angestrebt werden sollte. LS hatte keine Zeit für das Brainstorming und wird auch bis April 2023 wenig Zeit investieren. JZ (Smartwheel) erwähnte, dass es sinnvoll ist die Vision zu definieren. Nicht das die Kundenmeinungen uns umstimmen und wir den Fokus falsch setzten. ClickUp als Tool zur Taskerfassung damit wir von Anfang an kein chaotisches Startup sind. 

\section{Auswahl Idee}
Alle möchten die FlexBox weiterverfolgen.


\section{Weiteres Vorgehen Idee}
\begin{itemize}
\item In Bern gibt es ein Containerdorf. Wir sollten dieses demnächst besichtigen.
\item Recherche vertiefen, Marktrecherche
\item Internationalisierung
\end{itemize}


Im weiteren Schritt erfolgt: Jeder macht vertiefte Recherche. In einem zu definierenden Dokument wird die Analyse erarbeitet. 

\section{Bullet points für die Marktanalyse in PPTX}
Aufbau, Verkaufszahlenstatisk, Patent, Richtlinien, Logistik, Hardware (Wasser, HLK, Elektro), Software, Konstruktion, Innenausbau, Pro + Cons, Verbesserungsmöglichkeiten, Kunden + Kundenbedürfnis, Anwendungen, Modularitätsgrat (Wohnen, Behandeln, Informieren), Schnittstellen

\vspace{2cm}
\section*{Nächste Sitzung}

Montag, 26.09.2022, 20:00 Uhr, Meeting Online

\section*{Pendenzen}
\begin{tabularx}{17cm} { 
   C{1.0cm}
   L{7cm}
   C{2cm}
   C{2cm}
   C{2cm}
  }
 \toprule
Pendenz & Beschreibung & Zuständig & Datum & Status \\
\midrule
01	& PPTX aufstellen & JZ & 02.09.2022 & offen\\
02	& Marktrecherche ausfüllen  & alle & 02.09.2022 & offen\\
03	& Abklärung Besichtigung Containerdorf  & alle & 02.09.2022 & offen\\
\bottomrule
\end{tabularx}	
\par
\vspace{0.5cm}
\end{document}

